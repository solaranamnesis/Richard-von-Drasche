\documentclass[a4paper, 11pt, oneside]{article}
\usepackage[utf8]{inputenc}
\usepackage[T1]{fontenc}
\usepackage[ngerman]{babel}
\usepackage{yfonts}
%\usepackage{fbb} %Derived from Cardo, provides a Bembo-like font family in otf and pfb format plus LaTeX font support files
\usepackage{booktabs}
\setlength{\emergencystretch}{15pt}
\usepackage{fancyhdr}
\usepackage{graphicx}
\usepackage{microtype}
\graphicspath{ {./} }
\begin{document}
\frakfamily
\begin{titlepage} % Suppresses headers and footers on the title page
	\centering % Centre everything on the title page
	%\scshape % Use small caps for all text on the title page

	%------------------------------------------------
	%	Title
	%------------------------------------------------
	
	\rule{\textwidth}{1.6pt}\vspace*{-\baselineskip}\vspace*{2pt} % Thick horizontal rule
	\rule{\textwidth}{0.4pt} % Thin horizontal rule
	
	\vspace{1.5\baselineskip} % Whitespace above the title
	
	{\Huge Über den Meteoriten von Lancé.}
	
	\vspace{1.5\baselineskip} % Whitespace above the title

	\rule{\textwidth}{0.4pt}\vspace*{-\baselineskip}\vspace{3.2pt} % Thin horizontal rule
	\rule{\textwidth}{1.6pt} % Thick horizontal rule
	
	\vspace{1\baselineskip} % Whitespace after the title block
	
	%------------------------------------------------
	%	Subtitle
	%------------------------------------------------
	
	{\large Von Dr. Richard von Drasche.} % Subtitle or further description
	
	\vspace*{1\baselineskip} % Whitespace under the subtitle
	
	%------------------------------------------------
	%	Editor(s)
	%------------------------------------------------
    \vspace*{\fill}

	\vspace{1\baselineskip}

	{\small\scshape 1875.}
	
	\vspace{0.5\baselineskip} % Whitespace after the title block

    \scshape Internet Archive Online Edition  % Publication year
	
	{\scshape\small Namensnennung Nicht-kommerziell Weitergabe unter gleichen Bedingungen 4.0 International} % Publisher
\end{titlepage}
\setlength{\parskip}{1mm plus1mm minus1mm}
\clearpage
\LARGE
\pagestyle{fancy}
\fancyhf{}
\cfoot{\frakfamily{\thepage}}
\paragraph{}
Der Meteorit von Lancé ist seit vorigem Jahre durch Geschenk in den Besitz des k. k. mineralogischen Museums gekommen, nachdem sich seinerzeit sowohl der Eigentümer des Bodens, auf welchen er fiel, als die Gemeinde und der Finder den Besitz desselben streitig machten.

Über die Erscheinungen bei seinem Falle existieren zuverlässige und ausführliche Beobachtungen, so dass in dieser Beziehung dieser Meteorit als einer der genaueste bekannten bezeichnet werden kann.

In Folgendem ist die über diesen Meteoriten erschienene Literatur zusammengestellt:
\large
\begin{itemize}
    \item Chute d’un aérolithe dans la commune de Lancé, canton de Saint-Amand (Loir-et-Cher). Note de M. de Tastes, présentée par M. Ch. Sainte-Claire Deville. Comptes rendus 1872. Juillet p. 273.
    \item Note sur la découverte d’une seconde météorite tombée le 23 Juillet 1872, dans le canton de Saint-Amand (Loir-et-Cher) par M. Daubrée Compt. rend. 1872, Aout, p. 308.
    \item Examen des météorites tombées le 23 Juillet 1872 à Lancé et à Authon (Loir-et-Cher); par M. Daubrée Compt. rend. 1872, Aout, p. 465.
    \item Note additionelle sur la chute de météorites qui a eu lieu le 23 Juillet 1872, dans le canton de Saint-Amand (Loir-et-Cher); par M. Daubrée. Compt. rend. 1874, Aout, p. 277.
    \item Notice sur le bolide du 23 Juillet 1872, qui a projeté des météorites dans le canton de Saint-Amand, arrondissement de Vendôme, département de Loir-et-Cher par M. Nouel. Vendôme 1873.
\end{itemize}
\LARGE
\paragraph{}
Es sei uns erlaubt in kurzen Worten die Erscheinungen beim Falle des Meteoriten zu erwähnen; eine sehr weitschweifige Beschreibung und Zusammenstellung aller hierher gehörigen Umstände findet man in der oben zitierten Brochure von Nouel.

Um 5 Uhr 20 Minuten Nachmittags am 23. Juli 1872 bemerkte ein Beobachter zwischen Champigny und Brisay im Canton Saint-Amand, arrondissement de Vendôme, am Himmel einen Feuerstreif, der sich von Südwest nach Nordost bewegte und welcher sich plötzlich in zwei gesonderte Teile zu trennen schien. 6 Minuten nach der Wahrnehmung dieser Erscheinung wurde von dem Beobachter ein kanonenschuss-ähnlicher Schlag in der Umgebung von Tours vernommen; zur selben Zeit wurden auch in Tours zwei leuchtende Körper am Himmel gesehen.

Wenige Tage darauf fand man bei Lancé in einem Acker einen großen Meteoriten, welcher 1 M. 50 Cm. tief in dem Boden eingesunken war. Er war durch den Fall in 3 Teile zerbrochen.

Kurze Zeit nach diesem Funde entdeckte man in der Commune Authon, 2 Kilometer vom Orte, auf einem Platze, Pont Loisel genannt, einen anderen kleineren Meteoriten von genau derselben Beschaffenheit wie der von Lancé und sicher demselben Falle angehörig.

Der Punkt, wo dieser zweite Meteorit gefunden wurde, liegt 12 Kilometer südwestlich von demjenigen, wo der erstere fiel. Diese zwei Punkte liegen so ziemlich in einer Linie mit Champigny, wo zuerst die Feuererscheinung beobachtet wurde, und dürfte erstere mithin annähernd die horizontale Projektion der Meteoritenbahn auf die Erdoberfläche darstellen.

Im Jahre 1874 wurden neuerdings in derselben Gegend vier kleinere Meteoriten entdeckt, welche auch demselben Falle zuzuschreiben sind.

Die Gewichte dieser 6 Meteoriten in Kilogrammen ausgedrückt verhalten sich folgendermaßen: 47 der Meteorit von Lancé, 0.25 der von Authon und 3.00, 0.620, 0.600, 0.300 die vier zuletzt gefundenen.

Der Meteorit von Lancé ist, wie schon früher erwähnt wurde, beim Auffallen in drei Teile zersprungen, welche sich jedoch ganz genau wieder zusammenfügen lassen.

Die Form des Meteoriten ist die einer abgestumpften, vierseitigen Pyramide ähnlich. Berücksichtigt man die Zeichnungen auf der Oberfläche des Meteoriten, so muss man die Abstumpfungsfläche als Brustseite, die Basis der Pyramide als Rückseite betrachten. Die Brustseite ist beiläufig ein Trapez, dessen zwei längere Kanten 23 und 26 Cm. messen, die zwei kürzeren 18 und 12 Cm. Die Kanten der Pyramide sind alle sehr stark abgerundet.

Tafel 1 ist eine Ansicht des Meteoriten, der dem Beschauer die Brustseite zuwendet. Die in dieser Figur auf der unteren Hälfte liegenden Kanten sind am meisten abgerundet. Die Länge dieser vier Kanten beträgt: 16, 17, 18, 20 Cm.

Die Kanten der Pyramidenseiten mit der Basis sind scharf; die Basis selbst besteht aus zwei, unter einem Winkel von beiläufig 140° geneigten Flächen. Die Seiten der Pyramide machen mit der Brustfläche Winkel von 120-130°. Die Brustseite sowie die Seiten der Pyramide sind mit einer schwachen, schwarzen Schmelzrinde bedeckt, welche an vielen Stellen die graue Farbe des Meteoriten durchscheinen lässt.

Von der Mitte der Brustseite aus laufen sehr feine Linien, durch Anhäufung von Schmelzrinde erzeugt, strahlenförmig aus und konvergieren so in einem Punkte, von dem aus die flüssige Gesteinsoberfläche durch den Luftwiderstand nach den Seiten geblasen wurde. Die feinen Linien sind ebenfalls auf den Pyramidenseiten zu verfolgen. Hier werden sie oft senkrecht durch deutliche, sehr scharfe Linien abgeschnitten, längs welchen eine bedeutende Anhäufung von Schmelzsubstanz stattfindet. Solche Linien sind oft 2-3 hintereinander. Auf Tafel 3, Fig. 1 ist ein Teil einer Pyramidenseite dargestellt, um diese auffallenden Linien zu zeigen. Die Linien mögen durch eine schwingende Bewegung des Meteoriten um seinen Schwerpunkt während des Fluges entstanden sein.

Brust- und Seitenflächen zeigen keinerlei Vertiefungen, nur bei b) Tafel 1 gewahrt man drei tiefe rundliche Eindrücke, welche von einer starken Schmelzrinde umwallt sind. An derselben Fläche finden sich auch bei a) Tafel 1 mehrere längliche, ziemlich tiefe Eindrücke, durch welche ein Sprung geht. Gegen den Rand der Rückseite zu ist diese Fläche mit breiten, sehr schwachen Vertiefungen versehen, deren Richtung ziemlich parallel der Pyramidenkante verläuft und welche mit jenen Eindrücken zu vergleichen sind, welche entstehen, wenn man mit den Fingern über plastischen Ton fährt.

Tafel 2 stellt den Meteoriten von der Rückseite dar. Dieselbe ist von einer 0.5 Mm. dicken, schwarzen, leider etwas beschädigten Rinde bedeckt.

Die zwei Flächen, aus denen die Rückseite besteht, sind mit Ausnahme einer Stelle bei a) Fig. 2 vollkommen eben. Eine radiale Anordnung von Schmelzlinien vom Mittelpunkt der Basis aus nach den Seiten ist an manchen Stellen bemerkbar. Eine Erscheinung, welche noch die Rinde der Rückseite bietet, ist die, dass dieselbe wie von feinen Nadelstichen durchlöchert erscheint.

Diess dürfte von einem Entweichen von Gasen durch die noch weiche Schmelzrinde herrühren.

Der Bruch des Meteoriten ist feinkörnig und uneben, die Farbe frischer Bruchflächen ist grau. Schon mit freiem Auge erkennt man an ihnen die globuläre Struktur des Meteoriten, der zu Roses Klasse der Chondrit zu stellen ist. Die Kügelchen erreichen selten einen Millimeter im Durchmesser, sie sind entweder weiß oder dunkelgrau bis schwarz und reichlich. Außerdem beobachtet man noch eine große Anzahl von weißen, oft durchscheinenden Körnern mit deutlicher Spaltbarkeit, welche, wie es später die mikroskopischen Untersuchungen lehren werden, Olivin sind. Außer diesen zweierlei Kügelchen sind in der tufartigen Grundmasse noch häufig Partikelchen mit metallischem Glanze zu beobachten.

Das spezifische Gewicht des Steines ist nach Daubrée 3.80.

Nach den Erörterungen über die äußere Form und Beschaffenheit unseres Chondriten schreiten wir nun zu den Ergebnissen der mikroskopischen Untersuchungen, zu welchem Behufe Dünnschliffe angefertigt wurden.

In einer dunklen unentwirrbaren Grundmasse liegt eine außerordentliche Anzahl von kreisförmigen Durchschnitten von verschiedener Struktur, nebst einzeln zerstreuten Kristallfragmenten.

Auch dieser Chondrit entspricht mithin der Definition, welche G. Tschermak von denselben aufstellt: (Sitzungsberichte der Wiener Akademie, 1874, November. Über die Trümmerstruktur der Meteoriten von Orvinio und Chantonnay). „Chondrit sind mehr oder weniger tuffähnliche Massen, bestehend aus Gesteinskügelchen und einer pulvrigen und dichten, gleich zusammengesetzten Grundmasse.“

Die einzelnen Kügelchen in unserm Chondriten sind von großer Verschiedenheit und oft von so eigentümlicher Struktur, dass sie einer genaueren Beschreibung wert erscheinen.

a) Kügelchen, meistens mit schön kreisförmigen Durchschnitten, weiß, durchscheinend, sie bestehen meist aus einer großen Anzahl scheinbar unregelmäßig angeordneter Kriställchen, oft jedoch auch aus wenigen symmetrisch um einen Punkt gestellten Kristallen von weißer Farbe, durchsichtig und mit deutlicher Spaltbarkeit. Ich zweifle nicht daran, dass diese Kristalle Olivin sind, und die Olivin-Kügelchen den schon mit freiem Auge am Meteoriten beobachteten weißen Körnchen entsprechen. Fig. 1 auf Tafel 4 zeigt ein solches Olivin-Kügelchen in der dunklen Grundmasse eingebettet. Mit Anwendung von Polarisation beobachtet man, dass die drei Teile im Durchschnitte auch verschiedenen Individuen angehören.

Fig. 2, Tafel 4 zeigt ein aus vier größeren Individuen nebst einigen kleinen zusammengesetztes Kügelchen. Die einzelnen Individuen bilden auch hier im Durchschnitte Kreissegmente.

Eine Anzahl von undurchsichtigen, kugelförmigen Körperchen sind sowohl am Rande als in der Mitte des Durchschnittes zu beobachten.

Fig. 3, Tafel 4 zeigt den Durchschnitt eines etwas unregelmäßig begrenzten Kügelchens, welches aus dicht aneinander gestellten Polygonen besteht und im Mikroskope unwillkürlich an ein facettiertes Fliegenauge erinnert. Die einzelnen Kristalle, welche ganz unregelmäßige optische Orientierung zeigen, sind ebenfalls dem Olivin angehörig.

In Fig. 4, Tafel 4 ist der Durchschnitt eines ganz merkwürdigen Kügelchens abgebildet, von welcher Gattung ich nur ein einziges Exemplar beobachten konnte. Der Durchschnitt ist vollkommen kreisförmig, die Substanz, aus der das Kügelchen zum größten Teile besteht, ist farblos, sie zeigt jedoch bei gekreuzten Nicols keine weiteren Erscheinungen. Von einem exzentrisch liegenden Punkte strahlen sechs lanzettförmige Leistchen unter Winkeln von 45° nach den Rändern aus; an dieselben heften sich wieder andere kürzere Stäbchen, ebenfalls unter 45°, in großer Menge. Bei sehr starker, 240facher Vergrößerung, erscheinen dieselben hohl und teilweise mit einer dunkelgrünen, flockigen Substanz erfüllt.

Die gleichförmige Grundmasse des Kügelchens ist von vielen Sprüngen durchsetzt, welche ungehindert durch die Leistchen fortsetzen.

Ein anderes merkwürdiges, ebenfalls nur in einem Exemplare in unseren Dünnschliffen vorkommendes Kügelchen stellt Fig. 5 auf Tafel 4 dar. Es hat einen Durchmesser von 1 Mm. und ist schon mit freiem Auge deutlich am Präparate sichtbar. Der Durchschnitt ist schön kreisförmig und besteht aus zwei Teilen, einem inneren Kern und einem äußeren Ring von der Breite 1/3 radius der Kugel. Der innere Teil ist ungefähr von Kreisform, wird aber, genau genommen, meistens von graden, oft unter spitzen Winkeln zusammenstoßenden Linien begrenzt. Er ist mit einer dunkelbraunen, undurchsichtigen, gegen polarisiertes Licht sich passiv verhaltenden Masse erfüllt, welche hie und da Anlage zur blätterigen Ausbildung zeigt.

In diese Masse sind viele kleine, stark polarisierende, farblose Körperchen eingebettet, welche ich für Olivin halten möchte.

Von den Ecken an der Oberfläche dieses inneren Teiles gehen starke, gekrümmte Adern nach dem Rande des äußern Teiles und teilen so den Ring in eine Anzahl Sektoren. Der äußere Ring selbst besteht wieder aus einem Aggregat der kleinen farblosen Kristallischen, welches von einem dichten Netzwerk eines braunen, faserigen Minerals durchzogen ist. Ich vermute, dass dasselbe aus derselben Substanz bestehe, wie der innere Teil der Kugel.

b) Während die sub a) aufgezählten Kügelchen größtenteils aus Olivin bestanden, kommen wir jetzt zu der Beschreibung von Kugeln, welche aus einem feinfaserig, exzentrisch angeordneten Minerale bestehen, welches wohl in den meisten Fällen Bronzit sein dürfte. Schon G. Rose (Beschreibung und Einteilung der Meteoriten) hat, mit damals noch unzureichenden mikroskopischen Hilfsmitteln diese Art von Kügelchen beschrieben und abgebildet und hauptsächlich im Gegensatze zu terrestrischen ähnlichen Gebilden die stets exzentrische Struktur derselben hervorgehoben. Von G. Tschermak besitzen wir genaue und ausführliche Beschreibungen dieser Körper in dem Meteorit von Gopalpur. (Die Meteoriten von Shergotty und Gopalpur. 65. Bd. der Sitzb. der 5 Akademie der Wissensch., 1. Abt., Februar-Heft, Jahrgang 1872.)

Fig. 7, Tafel 4 ist die Abbildung eines exzentrischen Kügelchens aus unserem Meteoriten. Dasselbe ist ungemein dickfaserig, so dass es selbst im Dünnschliffe nur schwach Licht durchlassend ist.

In Fig. 6 ist ein anderes Kügelchen abgebildet, ebenfalls mit exzentrischer Anordnung von einem Punkte des Randes. Die einzelnen Radien lösen sich bei sehr starker Vergrößerung in Flöckchen auf, so dass es den Anschein hat, als wären dieselben eher einer radial angeordneten interponirten Substanz als einer Folge innerer Struktur zuzuschreiben. Eine Beobachtung bei gekreuzten Nicols lehrt jedoch augenblicklich durch die verschiedene radiale optische Stellung der einzelnen Sektoren, dass wir es in der Tat mit einer radialen Struktur zu tun haben.

Fig. 8 zeigt uns den Durchschnitt eines faserigen Kügelchens, welcher wohl senkrecht zur Längsausdehnung der Fasern geschliffen ist.

Außer diesen faserigen Kügelchen, welche in großer Menge in unserem Meteoriten vorkommen, beobachtete ich ein Kügelchen, welches nur aus einem Gewirre von Bronzit-Kristallen besteht. (Siehe Fig. 9, Tafel 4 in 240facher Vergrößerung.) Die einzelnen Kristalle sind zwar so unendlich klein, dass eine Bestimmung ihrer optischen Hauptschnitte unmöglich ist, jedoch die lange nadelförmige Gestalt, die Zerteilung der einzelnen Kristalle durch Quersprünge deuten unbedingt auf ein Mineral der Bronzitgruppe hin. Manche Nadeln sind von ungeheurer Dünne, andere erreichen wieder verhältnismäßig ansehnliche Breite, stets sind sie aber ohne jedes Gesetz zu einander gruppiert.

Meines Wissens wurde eine ähnliche Kugel noch nie in Meteoriten beobachtet.

Wir haben nun noch schließlich die im Chondriten von Lancé einzeln vorkommenden Mineralien zu besprechen. Es sind dies Eisen, Magnetkies, Bronzit, Olivin.

Eisenkies und Magnetkies lassen sich bei auffallendem Lichte leicht durch ihre verschiedenen Farben erkennen. Beide sind in großer Menge in unserem Meteoriten zerstreut. Überall, sowohl in der tufähnlichen Grundmasse, als in den Kügelchen und einzelnen Kristallen trifft man diese Mineralien in großer Häufigkeit an. Teils kommen beide isoliert vor, teils beobachtet man größere unförmlich kugelige Massen, die einen Kern von Magnetkies und eine Hülle von Eisen oder umgekehrt zeigen.

Ob Chromeisen auch vorhanden ist, konnte ich nicht beobachten, die Analyse von Daubrée macht diess jedoch sehr wahrscheinlich.

Einzelne Olivin-Kristalle von ansehnlicher Größe bis 1 Mm. kommen sehr häufig vor. Sie zeigen oft ziemlich regelmäßige, geradlinige Begrenzung, sind farblos-durchsichtig im Schliffe und von den dem Olivin eigentümlichen Sprüngen zahlreich durchsetzt. (Siehe Fig. 10, Tafel 4.)

In allen unseren Dünnschliffen konnten wir nur einen einzigen isolierten größeren Bronzit-Kristall beobachten. (Siehe Fig. 11, Tafel 4.) Derselbe ist in der dichten Grundmasse eingebettet und zeigt sehr deutliche Spaltbarkeit. Die Spaltungsdurchgänge sind mit einer undurchsichtigen Substanz erfüllt. Die optischen Hauptschnitte stehen senkrecht zu der Spaltungs- und Längsrichtuug des Kristalles; es kann mithin kein Zweifel an der rhombischen Natur dieses Kristalles sein. Der Kristall selbst ist durch mechanische Gewalt, wie es scheint, bedeutend zerstickt und zerquetscht.

Unsere mikroskopischen Beobachtungen können wir nunmehr mit folgenden Worten kurz zusammenfassen: In einem tufartigen Zerreibselliegen viele isolierte Kristalle von Olivin und hie und da Bronzit, nebst einer großen Menge von Kügelchen von zweierlei Beschaffenheit. Dieselben sind entweder regelmäßige oder unregelmäßig angeordnete Aggregate von Olivin, oder bestehen aus exzentrisch-strahlig angeordneten Bronzit-Nadeln.

In einem speziellen Falle bestand eine Kugel aus einem wirren Haufwerk von Bronzit-Kristallen. Magnetkies und Eisen sind reichlich in allen Teilen des Chondriten vorhanden.

Was schließlich die chemische Zusammensetzung des Chondriten von Lancé betrifft, so besitzen wir eine Analyse desselben von Daubrée und ich erlaube mir die darauf bezüglichen Stellen folgend in Übersetzung wiederzugeben: (Siehe Examen des météorites tombées le 23. Juillet 1872, par M. Daubrée Compt. rend. Aout 1872. pag. 467)

„Mit Wasser behandelt verliert die Substanz 0.12\%, Chlornatrium....

Wenn man die Substanz der Rotbglühhitze in einem Strome von Wasserstoff aussetzt und das erzeugte Sublimat auffängt, so kann man von Neuem die Gegenwart des Chlornatriums in demselben Verhältnisse konstatieren, als es in der wässerigen Lösung gefunden wurde. Kalisalze, Sulfate und Hypersulfate sind nicht vorhanden. Salzsäure und Schwefelsäure bewirken eine Entwicklung von Schwefelwasserstoffgas in großer Menge, aber ohne einen Absatz von Schwefel, welches anzeigt, dass sich der Schwefel nur als Protosulfür vorfinde. Man hat sowohl die Menge des Schwefels des entwichenen Schwefelwasserstoffgases mit salpetersauren Silberoxid bestimmt, als auch die Menge des entwichenen Wasserstoffes von der Behandlung mit Schwefelsäure herrührend, und es ist durch letztere Methode gelungen, den approximativen Gehalt an freien Metallen zu bestimmen. Durch Behandlung mit Salpetersäure und indem man nach der Methode von H. Sainte-Claire Deville vorging, hat man die Gegenwart eines angreifbaren Silikates konstatiert, welches Magnesia und Eisenoxydul enthält.

Der unangreifbare Teil besteht aus wenigstens zwei Substanzen, einem farblosen und einem tiefschwarzen.

Das olivinähnliche Silikat beträgt 42.36\%, des Totalgewichtes, der unzersetzte Teil 33.44%.

Das Eisen aus dem in Salzsäure löslichen Teil wurde nach der Margueritte'schen, durch Boussingault verbesserten Methode bestimmt; es beträgt 24.48%.

Die Gegenwart des Kupfers wurde durch die Spektral-Analyse erkannt, ebenso die Abwesenheit von Kalk, Baryum und Strontium. Kohle konnte nicht nachgewiesen werden. Wie gewöhnlich begleiten Kobalt und Nickel das Eisen in diesem Meteoriten.

Folgendes ist das Resultat der Analyse:
\normalsize
\begin{center}
\begin{tabular}{ l r }
    Freies, mit Nickel und Kobalt legiertes Eisen & 7.81\\
    Eisen u. andere Metalle an Schwefel gebunden (Protosulfür) & 9.09\\
    Gebundener Schwefel (Protosulfür) & 5.19\\
    Durch Säuren zersetzbares Silikat oder Olivin (Kieselsäure) & 17.20\\
    Durch Säuren zersetzbares Silikat oder Olivin (Magnesia) & 13.86\\
    Durch Säuren zersetzbares Silikat oder Olivin (Eisenoxydul) & 11.33\\
    Durch Säuren zersetzbares Silikat oder Olivin (Manganoxydul) & 0.05\\
    Durch Säuren unzersetzbarer Teil & 33.44\\
    Chlornatrium & 0.12\\
    Hygroskopisches Wasser & 1.24\\
     & 99.31\\
\end{tabular}
\end{center}
\LARGE
\paragraph{}
Als Bestätigung füge ich hinzu, dass sukzessive Ströme von Wasserstoff und Chlor eine Gewichtsabnahme von 34.98\% bewirkten. Vergleicht man diese Ziffer mit denen der Analyse, so kommt man zur Überzeugung, dass nach dieser Operation nur mehr das unzersetzbare Silikat und die Kieselsäure und Magnesia des zersetzbaren Teiles zurückbleiben.

Abgesehen von den gewöhnlichen Bestandteilen eines Meteoriten wie Nickeleisen, Troilit, Olivin und unzersetzbares Silikat, enthält der Meteorit von Lancé Chlornatrium in kleiner Menge.“

Wir erkennen somit auch in der chemischen Analyse die Mineralien wieder, welche wir im Mikroskope beobachten konnten. Der unzersetzbare, nicht weiter analysierte Teil besteht nach Daubrée aus einem farblosen und schwarzen Mineral. Ersteres ist wohl Bronzit, letzteres dürfte Chromit sein.

Die 14.28\%, Protosulfür werden wohl dem Magnetkies angehörig sein, da kein Troilit von uns beobachtet wurde. Zählen wir nunmehr die beobachteten Mineralien auf, so enthält der Chondrit von Lancé Nickeleisen, Magnetkies, Chromit, Olivin und Bronzit. Troilit wurde nicht beobachtet. Eine genaue Analyse des unlöslichen Teiles müsste zeigen, ob in demselben nicht vielleicht auch wie im Meteoriten von Gopalpur ein feldspatähnlicher Bestandteil vorhanden ist.

Zum Schlusse entledige ich mich einer angenehmen Pflicht, wenn ich Herrn Direktor Dr. G. Tschermak meinen verbindlichsten Dank ausdrücke für die Liberalität, mit welcher er mir die hiesige Meteoritensammlung zu meinen Studien zur Verfügung stellte.
\clearpage
\setlength\intextsep{0pt}
\pagestyle{fancy}
\fancyhf{}
\rhead{\frakfamily{Tafel 1 und 2.}}
\cfoot{\frakfamily{\thepage}}
\begin{figure}[p]
\includegraphics[scale=0.6,keepaspectratio]{Figures/table1.png}\tiny 1.1
\includegraphics[scale=0.6,keepaspectratio]{Figures/table2.png}\tiny 2.1
\end{figure}
\clearpage
\rhead{\frakfamily{Tafel 3.}}
\begin{figure}[p]
\includegraphics[scale=0.7,keepaspectratio]{Figures/table3-figure1.png}\tiny 3.1
\includegraphics[scale=0.9,keepaspectratio]{Figures/table3-figure2.png}\tiny 3.1
\end{figure}
\clearpage
\rhead{\frakfamily{Tafel 4.}}
\begin{figure}[p]
\includegraphics[scale=0.9,keepaspectratio]{Figures/drasche-figure3final.png}
\end{figure}
\end{document}
